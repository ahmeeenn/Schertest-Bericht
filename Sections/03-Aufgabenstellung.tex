<<<<<<< HEAD
Ziel dieses Versuchs ist es, die mechanischen Eigenschaften von Kupfer-Scherkörpern zu untersuchen und zu vergleichen:
\\
\vspace{0.05cm}
\begin{figure}[h]
    \centering
    \includegraphics[scale=0.95]{Bilder/schematik.png}
    \caption{Versilberter Kupfer Scherkörper gesintert auf Kupferbodenplatte}
<<<<<<< Updated upstream
    \caption*{\textit{Quelle: https://tinyurl.com/2p8ejcnv}}
=======
    \par Quelle: \href{https://learn.fh-kiel.de/mod/folder/view.php?id=160797}{Versuch 1 Schertest.pdf}
>>>>>>> Stashed changes
    \vspace{0.2cm}
    \label{Abb.2: Versilberter Kupfer Scherkörper gesintert auf Kupferbodenplatte} 
\end{figure}\\

\begin{itemize}
\item Durchführung von Schertests an Prüflingen, wobei die Scherkörper durch Schubspannung mittels eines Schermeißels abgeschert werden.
\item Nach dem Scheren werden die einzelnen Prüfkörper auf Karteikarten aufgeklebt und mit dem zugehörigen, aus der Prüfsoftware abgelesenen Kraftwert beschriftet.
\item Die aufgenommenen Kraftwerte für alle Prüfkörper in einem Boxplot-Diagramm dargestellen.
\item Nach Abschluss der Scherversuche eine mikroskopische Untersuchung der Bruchstellen.
\item Im Fokus steht dabei die Identifikation und Analyse der verschiedenen Bruchmuster.
\item Das Ziel besteht darin, Unterschiede im Haftverhalten sowie in der Bruchcharakteristik der verwendeten Prüflinge zu erkennen und zu bewerten.
\item Gemessene Kraftwerte in Spannungen $\mathrm{N/mm^2}$  umrechnen.
\end{itemize}
=======
Ziel dieses Versuchs ist es, die mechanischen Eigenschaften von Kupfer-Scherkörpern zu untersuchen und zu vergleichen:\\
\subsection{Untersuchung der Prüfkörper}
Hiermit werden zwei Kupfer-Scherkörpern untersucht mit jeweils eine Art:\\
Scherprüfling 1: Versilberter Kupfer Scherkörper gesintert auf Kupferbodenplatte.\\
\vspace{0.05cm}
\begin{figure}[h]
    \centering
    \includegraphics[scale=0.1, angle=90]{Bilder/Bodenplatte_Sintern_Gesamt.jpg}
    \caption{Versilberter Kupfer Scherkörper gesintert auf Kupferbodenplatte, (Selbsterstellte Abbildungen)}
    \vspace{0.2cm}
    \label{Abb.2: Versilberter Kupfer Scherkörper gesintert auf Kupferbodenplatte} 
\end{figure}\\
Scherprüfling 2: Kupfer Scherkörper laminiert auf Kupferbodenplatte
\vspace{0.1cm}
\begin{figure}[h]
    \centering
    \includegraphics[scale=0.06, angle=90]{Bilder/Laminieren_Bodenplatte_Gesamt.jpg}
    \caption{Kupfer Scherkörper laminiert auf Kupferbodenplatte, (Selbsterstellte Abbildungen)}
    \vspace{0.2cm}
    \label{Abb.3: Kupfer Scherkörper laminiert auf Kupferbodenplatte}
\end{figure}
\newpage
\subsection{Durchführung}
Jeder Prüfling besitzt in seiner Platte zehn Scherkörper, die durch Schubspannung mit einem Schermeißel abgeschert werden.
\vspace{0.1cm}
\begin{figure}[h]
    \centering
    \includegraphics[scale=0.4]{Bilder/Schermeißel.png}
    \caption{Schertester Condor Sigma des Herstellers XYZTec, (Selbsterstellte Abbildungen)}
    \vspace{0.2cm}
    \label{Abb.4: Schertester Condor Sigma des Herstellers XYZTec}
\end{figure}
\vspace{0.1cm}
Jeder der Prüfkörper wird nach dem Scheren auf eine Karteikarte geklebt, wobei der jeweilige Kraftwert eingetragen wird.
Dieser Wert wird aus der Computersoftware abgelesen.
\subsection{Untersuchung}
Nach Durchführung der Scherversuche werden die Bruchstellen der Prüfkörper mithilfe eines Digitalmikroskops untersucht und festgehalten.
Dabei steht die Identifizierung und Analyse der entstandenen Bruchmuster im Fokus. 
>>>>>>> d82e7771a987df4bf3ee26cab5f2ec0e9c3fcce8
