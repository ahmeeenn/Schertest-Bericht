Der Versuch wurde am Freitag den 21.03.2025 durchgeführt und wurde in einem Zeitraum von 90min abgeschlossen.
