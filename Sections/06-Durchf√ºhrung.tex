Die Versuchsdurchführung erfolgt gemäß der Norm MIL-STD-883E. Der Messplatz befindet sich im Gebäude C12, Raum 3.28. Die verwendeten Geräte umfassen den Schertester Condor Sigma von XYZTec sowie das Mikroskop Keyence VHX 1000. Das Experiment wird in zwei Abschnitten durchgeführt: zunächst der Schertest, gefolgt von der Analyse der Bruchbilder.

\subsection{Schertest Durchführung}
Die Proben werden gemäß den Spezifikationen vorbereitet. Es werden zwei Scherprüflinge untersucht:
\begin{itemize}
    \item Scherprüfling 1: Versilberter Kupfer-Scherkörper, gesintert auf eine Kupferbodenplatte.
    \item     Scherprüfling 2: Kupfer-Scherkörper, laminiert auf eine Kupferbodenplatte.
\end{itemize}