Der durchgeführte Schertest basiert auf der Norm MIL-STD-883E und untersuchte zwei verschiedene Scherprüflinge:
\begin{enumerate}
    \item Scherprüfling 1: Versilberte Kupferscherkörper, gesintert auf einer Kupferbodenplatte
    \item Scherprüfling 2: Kupferscherkörper, laminiert auf einer Kupferbodenplatte.
\end{enumerate}
Jeder Scherkörper bestand aus acht Scherprüflingen, die mithilfe eines Klebstoffs auf der Kupferbodenplatte fixiert wurden.
Für den Schertest wurden folgende Prüfgeräte verwendet:
\begin{itemize}
    \item Schertester Condor Sigma der Firma XYZTec
    \item Digitalmikroskop zur hochauflösenden Analyse der Bruchflächen
\end{itemize}
Für dieses Schertest ist sowohl ein Schertester Condor Sigma von der Firma XYZTec als auch ein Digitalmikroskop  zu nutzen. 
\subsection{Schertester Condor Sigma von XYZTec}
Der Schertester besteht aus einem Schermeißels, ein Mikroskop und sein eigener Rechner. 
Der XYZTec Condor Sigma ist ein hochmoderner Schertester, der für präzise und automatisierte Prüfungen von Verbindungen in der Halbleiter und Elekronikindustrie entwickelt wurde:
Der Schetester besitzt die folgende Eigenschaften:
\begin{itemize}
    \item Modularer Aufbau: Der Condor Sigma verfügt über eine modulare Architektur, die es ermöglicht, verschiedene Testköpfe zu integrieren, darunter die Rotierende Messeinheit RMU mit bis zu sechs Sensoren. Dies erlaubt kontinuierliche Tests ohne manuelle Umrüstungen. 
    \item Automatisierung: Das System bietet umfassende Automatisierungsfunktionen, einschließlich roboterbasierter Handhabung für sicheres Laden und Entladen von Proben. Die offene Sigma-Software ermöglicht eine einfache Programmierung aller Automatisierungsschritte mithilfe von Kameravisualisierung und intelligenten Assistenten.
    \item Präzise Positionskontrolle: Ein integrierter dynamischer Motion-Controller und lineare Encoder verbessern die Positionsgenauigkeit und Reproduzierbarkeit in den X-, Y- und Z-Achsen erheblich.
    \item Hohe Messgenauigkeit: Der Nano-Control-Scherkraftsensor ermöglicht eine außergewöhnliche Präzision in der Scherhöhenmessung mit einer Genauigkeit von bis zu 200 nm. 
    \item Vielseitige Testmöglichkeiten: Der Condor Sigma kann für verschiedene Testarten wie Scher-, Zug- und Drucktests konfiguriert werden und deckt dabei einen Kraftbereich von weniger als 0,1 gf bis zu 10 kgf ab. 
    \item Hochauflösende Bildgebung: Das System unterstützt bis zu drei Live-Kameras mit flexibler LED-Beleuchtung und bietet umfangreiche Bildverarbeitungsoptionen für detaillierte optische Inspektionen. cite{2}
\end{itemize}
\subsubsection{Vorteile von dem XYZTec Condor Sigma}
\begin{enumerate}
    \item Flexibilität: Dank des modularen Designs kann der Condor Sigma an spezifische Prüfanforderungen angepasst werden, was eine hohe Vielseitigkeit in verschiedenen Anwendungen ermöglicht.
    \item Effizienzsteigerung: Die umfassende Automatisierung reduziert menschliche Fehler und senkt Produktionskosten durch schnellere und konsistentere Prüfprozesse.
    \item Benutzerfreundlichkeit: Die intuitive Softwareoberfläche und die einfache Programmierung erleichtern die Bedienung und verkürzen die Einarbeitungszeit für das Personal.
    \item Zukunftssicherheit: Durch kontinuierliche Verbesserungen und Updates bleibt der Condor Sigma auf dem neuesten Stand der Technik und erfüllt aktuelle sowie zukünftige Prüfanforderungen.cite{3}
\end{enumerate}
\subsection{Einsatz des Digitalmikroskops}
Während des Versuchs wurde ein hochauflösendes Digitalmikroskop verwendet, um die Bruchflächen detailliert zu analysieren. Dies ermöglichte eine präzise Auswertung der Bruchmechanismen und eine zuverlässige Beurteilung der Materialeigenschaften.
