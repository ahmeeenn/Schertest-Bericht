Der durchgeführte Schertest analysiert die mechanischen Eigenschaften gesinterter und laminierter Kupferscherkörper gemäß der Norm MIL-STD-883E. Die Versuchsdurchführung erfolgte mithilfe eines Schertesters sowie eines hochauflösenden Digitalmikroskops zur detaillierten Untersuchung der Bruchbilder. Ziel der Studie war die Bestimmung der maximalen und durchschnittlichen Scherkraft sowie der Scherfestigkeit der Prüfkörper, um die Unterschiede zwischen beiden Herstellungsverfahren systematisch herauszuarbeiten. Darüber hinaus diente der Versuch zur Evaluierung des Schertests, der verwendeten Messgeräte und des experimentellen Laborumfelds.\\
Neben der Bestimmung der Scherfestigkeit wurde die Streuung der Messergebnisse analysiert. Die individuellen Abweichungen der Prüfkörper wurden statistisch ausgewertet und in einem Boxplot visualisiert, um eine differenzierte Beurteilung der Messwerte und ihrer Variabilität zu ermöglichen.\\
Die Untersuchungsergebnisse zeigen signifikante Unterschiede zwischen gesinterten und laminierten Prüfkörpern. Gesinterte Scherkörper weisen eine höhere maximale Scherkraft und Scherfestigkeit auf, was auf eine festere und belastbarere Verbindung zwischen Scherkörper und Bodenplatte hindeutet. Die Bruchbildanalyse bestätigt diese Beobachtung durch charakteristische Unterschiede in den Bruchmechanismen.\\