Die durchgeführten Schertests verdeutlichen den erheblichen Einfluss der Materialwahl auf die mechanischen Eigenschaften der Prüfkörper. Gesinterte Scherkörper zeigen konsistent höhere Scherfestigkeiten und stabilere Bruchmechanismen im Vergleich zu laminierten Proben, was darauf hinweist, dass das Sintern eine festere und langlebigere Verbindung erzeugt als die Laminierung.\\
Es ist jedoch zu berücksichtigen, dass die Prüfkörper manuell auf die Bodenplatte aufgesetzt wurden, was potenziell Einfluss auf die Messergebnisse hatte. Diese manuelle Platzierung könnte zu Variationen in der Ausrichtung oder der Kontaktfläche geführt haben und somit die Messergebnisse beeinflusst haben. Eine standardisierte, automatisierte Positionierung könnte die Reproduzierbarkeit und Genauigkeit zukünftiger Untersuchungen weiter verbessern.\\
Die gewonnenen Messergebnisse liefern eine wertvolle Grundlage für industrielle Anwendungen, insbesondere in Bereichen mit hohen mechanischen Belastungen. Die Analyse der Bruchbilder bestätigt zudem die theoretischen Annahmen über die Bruchmechanismen der jeweiligen Herstellungsverfahren und bietet weiterführende Erkenntnisse für die Optimierung der Materialauswahl und Fertigungsprozesse.\\